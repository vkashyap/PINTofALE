\documentstyle[11pt]{article}
%
\setlength{\parindent}{1 cm}
\setlength{\parskip}{\baselineskip}
\setlength{\textwidth}{16.5 cm}
\setlength{\textheight}{22.5 cm}
\setlength{\topmargin}{0.0 cm}
\setlength{\oddsidemargin}{0.0 cm}
\setlength{\evensidemargin}{0.0 cm}
\setlength{\headsep}{0.0 cm}
%
\raggedright
%
\def\chisq{$\chi^2$}
\def\etal{{\sl et al.\/}\ }
\def\kms{$\,$km$\,$s$^{-1}$}
\def\sqig{$\sim$}
\def\sun{$_\odot$}
\def\ergs{$\,$erg$\,$s$^{-1}$}
\def\espc{$\,$erg$\,$s$^{-1}$\,$pc$^{-3}$}
%
\begin{document}
%
\

\vspace{0.5 cm}
\begin{centering}

{\Large\bf
PIMMS version 4.8a Users' Guide \\
}

\vspace{0.5 cm}

{\Large

K. Mukai \\

}

\end{centering}

\section{Introduction}

PIMMS (Portable, Interactive, Multi-Mission Simulator) is intended
as a versatile simulation tool for X-ray astronomers.

PIMMS uses one command {\tt GO} for actual execution, while other commands
are mostly used for setting up various parameters.  This approach allows
users to repeat similar calculations using a slightly different parameter.

PIMMS uses the following terms:

\begin{itemize}
\item Model: spectral model to be used.  PIMMS contains a small set of
simple spectral models, and others can be imported.
	\begin{itemize}
	\item However, spectral simulation is not a strength of PIMMS
	--- it does not output spectra as such, and it is unlikely that
	PIMMS can keep up with the multitude of models that are used for
	various types of objects.  We recommend
	the use of XSPEC for a full spectral simulation.
	\end{itemize}
\item Instrument: in addition to the instrument for which simulation is
to be done, you can also specify the {\bf input} instrument
for count rate calculations.
This is used for calculating normalization of the model; rather
than starting from the source flux in cgs unit, PIMMS can start from e.g.,
Einstein IPC count rate.  
\end{itemize}

\section{New in v4.8/4.8a}

Version 4.8a contains a slightly updated set of effective area curves
for Chandra, suitable for use for Cycle 18 proposals.

The pile-up formulae for the XMM-Newton EPIC instruments have been revised
using the latest information from the instrument teams. This brings PIMMS
in line with tools used at the XMM SOC.

It is released with new XMM-Newton EPIC effective area curves suyitable
for AO-15 proposal preparation, although any differences with the previous
version is exteremely minor.

It is also released with new ASTRO-H HXI effective area curves.

\section{New in v4.7/4.7a/4.7b/4.7c/4.7d}

Version 4.7d contains an updated, though still somewhat preliminary,
effective area curve for the NICER mission.

Version 4.7c is released with a new set of Chandra effective area curves
suitable for preparation of Cycle 17 proposals, and a new set of Suzaku
XIS effective area curves suitable for preparation of AO-10 proposals.
However, any differences with the respective previous versions are
extremely minor.

Version 4.7b is released with new XMM-Newton EPIC effective area curves
suitable for AO-14 proposal preparation, although any differences with
the previous version are extremely minor.

Version 4.7a contains an updated effective area curve and associated
information for NuSTAR, reflecting the post-launch understanding of the
mission.  Although documentations for version 4.7 (released 2014 March 5)
claimed this update was made in that version, this was regrettably not
the case due to a configruation management error. In version 4.7a, the
new effective area curve is includedn and the upper limit of the instrument
energy range was updated from 80 keV to 79 keV.

Version 4.7 newly includes effective area curves for MAXI SSC and GSC
all-sky monitor instruments. It also newly includes support for the HEAO-1
A1 instrument, appropriate for a direct comparison with the HEAO-1 A1 X-ray
Source Catalog. It also contains updated effective area curves for the Chandra
instruments, appropriate for Cycle 16 proposals.  Differences with Cycle 15
versions are minor.

The set of APEC model files have been re-generated using ATOMDB 2.0.2 using
xspec version 12.8.1g on 2013 Dec 9-11, with the ``solar'' abundance values of
Asplund et al. 2009.  Previous APEC model files were based on ATOMDB 1.3.1 and
Anders \& Grevesse solar abundances.

\section{New in v4.6/4.6a/4.6b}

Version 4.6b contains updated effective area curves for the Chandra
instruments that are suitable for Cycle 15 proposers, although differences
with the Cycle 14 versions are minor.  The array size for missions and
instruments have also been expanded in Version 4.6b, although this only
affects users who wish to add many extra missions/instruments to the
standard installation.

Version 4.6a contains a minor update of Suzaku XIS effective area curves
appropriate for AO-8 proposers.

Version 4.6 includes a preliminary effective area curve for the proposed
NICER mission.  It also includes a preliminary algorithm to estimate the
exposure times necessary to achieve 3 and 5 sigma detections with ASTRO-H
SGD, the accuracy of which is currently under investigation.

\section{New in v4.5}

Version 4.5 includes an updated set of effective area curves for Chandra.
It also includes a minor bug fix to the module that creates the help file
used by PIMMS.

\section{New in v4.4}

Version 4.4 includes the effective area curve and further mission-specific
support for NuSTAR.  It also includes the initial estimate of SXS pile-up
for the ASTRO-H mission, along with updated effective area curves for all
instruments.  It also includees minor updates for the effective area curves
for XMM-Newton and Suzaku.

\section{New in v4.3}

Version 4.3 includes a new mission-speicific comments regarding Integral
JEM-X: two units are opreational since 2010 October, but PIMMS continues
to provide count rate per unit of JEM-X.

\section{New in v4.2/v4.2a}

Version 4.2a includes a new set of Chandra effective area curves, suitable
for AO-13 proposers.  Of the various instrument configurations, the HRC-I
effective area curves changed the most.

Version 4.2 includes a new set of Suzaku effective area curves, suitable
for Cycle 6 proposers, and an accompanying update in the code to estimate
detection limits for the HXD detectors.  Also, references to count rates
for point sources observed at the HXD nominal position have been removed,
as this pointing position is no longer recommended.

\section{New in v4.1/4.1a/4.1b}

Version 4.1b includes an updated effective area curves for XMM (suitable
for AO-10 proposals) and for IXO, and an updated version of pms\_slmdl.f
which conforms strictly to the Fortran standard.  (One compiler flagged
the older version as an error; if it compiled, it worked correctly.)

By the request of XMM-Newton project at ESA, PIMMS version 4.1 reverts
to using a large extraction region and PATTERNS 0 through 12 (for MOS)/4 (pn)
in calculating EPIC count rates.  Version 4.1 incorrectly stated that the
PATTERNS used was 0--4 for both EPIC-MOS and EPIC-pn, which was corrected
in Version 4.1a.

Version 4.1 includes preliminary effective area curves for ASTRO-H,
the Japanese-US mission scheduled for launch in 2014.

Version 4.1 now reports energy/wavelength ranges with an increased
number of digits.

\subsection{New in v4.0}

As of version 4.0, two new grids of models for collisionally excited thermal
plasma (using mekal and APEC codes) are available.  The expanded grid of
Raymond-Smith plasma model that was available as an add-on to the previous
distributions are now standard; the more limited grid that was the standard
in V3.9k and earlier versions has now been retired.  Of the three grids,
APEC is now the default.  To swith among the plasma models, version 4.0
introduce a new command, ``PLASMA.''  The syntax of the ``MODEL'' command
has accordingly changed.  The old model name, ``raymond-smith'' (or RS) is
no longer recognized, to be replaced with ``PLASMA''.  The model component,
plasma, requires 3 numerical parameters: temperature, abuandance, and Nh,
in that order.  An optional string parameter to be placed after the
temperature can be used to switch the units of temperature from keV
(default) to logT.

Log files now contain command echos.

As of Version 4.0, the effective area curves for missions Constellation-X,
XEUS, and Spectrum X-Gamma have been removed from PIMMS.  Those for the
International X-ray Observatory (IXO) have been added.

Version 4.0 includes updated effective area curves for the Chandra
instruments, suitable for Cycle 12 proposers.

See Appendix B, ``Older Updates,'' for a log of additions and changes
before v4.0.

\section{Sample Sessions}

These sample sessions (using PIMMS v4.0) are avialable as *.xco files
in the `sample' subdirectory of PIMMS.

\subsection*{Example 1. Estimating Chandra ACIS-I count rate}

\begin{verbatim}
*** PIMMS version 4.0 ***
    2010 Feb 17th Release
    Reading mission directory, please wait
* Current model is BREMSSTRAHLUNG, kT=  10.0000 keV; NH =  1.000E+21
   <--- Use 'MODEL' command to change
        and 'PLASMA' command to switch among APEC/mekal/RS
* By default, input rate is taken to be
 Flux (    2.000-   10.000 keV) in ergs/cm/cm/s
   <--- Use 'FROM' command to change the default
* Simulation product will be
 Count rate in CHANDRA ACIS-I
   <--- Use 'INSTRUMENT' command to switch to another instrument
PIMMS > go 1 einstein ipc
* For thermal Bremsstrahlung model with kT= 10.0000 keV; NH =  1.000E+21
  and  1.000E+00 cps in EINSTEIN IPC
  (Internal model normalization =  5.882E-03)
* PIMMS predicts  4.518E+00 cps with CHANDRA ACIS-I
% Pileup estimate for ACIS:
  Pile-up is too high (30.9 %) at the fastest single-chip frame time (0.2 s)
  Consider using the Continuous Clocking mode
PIMMS > quit
\end{verbatim}

In this example, the default spectral model is used to estimate
the Chandra ACIS-I count rate (which happens to be the default). 
The only place where user did not use the default set-up was to specify
conversion from Einstein IPC count rate.  PIMMS not only reports
the total ACIS-I count rate but also the pile-up fraction for such
a source.

\subsection*{Example 2. Estimating XTE count rates I}

\begin{verbatim}
*** PIMMS version 4.0 ***
    2010 Feb 17th Release
    Reading mission directory, please wait
* Current model is BREMSSTRAHLUNG, kT=  10.0000 keV; NH =  1.000E+21
   <--- Use 'MODEL' command to change
        and 'PLASMA' command to switch among APEC/mekal/RS
* By default, input rate is taken to be
 Flux (    2.000-   10.000 keV) in ergs/cm/cm/s
   <--- Use 'FROM' command to change the default
* Simulation product will be
 Count rate in CHANDRA ACIS-I
   <--- Use 'INSTRUMENT' command to switch to another instrument
PIMMS > from exosat me
PIMMS > inst xte pca
PIMMS > mo plasma 1.0 1.0 5e19
NOTE: This version of PIMMS has a grid of  59x 5 grid of APEC models
      from kT= 0.034 keV (logT= 5.60) to kT=27.250 keV (logT= 8.50)
      and abundances from 0.20 to 1.00
         Selected temperature is  0.967 keV (log T is  7.05)
         and selected abundance is 1.0
PIMMS > go 1
* For PLASMA (APEC) model with
                        kT= 0.9669keV (logT= 7.05), Abund=1.0; NH =  5.000E+19
  and  1.000E+00 cps in EXOSAT ME
  (Internal model normalization =  3.273E-02)
* PIMMS predicts  2.505E-01 cps with XTE PCA
   (Count rate is per PCU)

%%%        With 3 PCUs operational:
           (Use these numbers in RPS)

PIMMS predicts     0.751 cps from the source plus    91.380 background cps
5-sigma detection will be achieved in 4078.777s
(but undetectable with 1% systematic uncertainties in bgd)

Results in the 6 canonical XTE PCA bands are:

Channels  Nominal    Source   BGD   5-sigma    (+1%)
          E (keV)    (cps)    (cps) detection (s)
  0- 13  0.00- 6.14     0.711 10.56  557.232 ( 1241.915)
 14- 17  6.14- 7.90     0.034  3.46 7.47E+04 (*********)
 18- 23  7.90- 10.5  5.75E-03  4.56 3.46E+06 (*********)
 24- 35  10.5- 15.8  3.93E-04  7.87 1.28E+09 (*********)
 36- 49  15.8- 22.1  1.80E-06  8.90 6.83E+13 (*********)
 50-249  22.1-116.0  3.19E-09 56.05 1.38E+20 (*********)

%%% ...and with 2 PCUs operational:

PIMMS predicts     0.501 cps from the source plus    60.920 background cps
5-sigma detection will be achieved in 6.12E+03s
(but undetectable with 1% systematic uncertainties in bgd)

Results in the 6 canonical XTE PCA bands are:

Channels  Nominal    Source   BGD   5-sigma    (+1%)
          E (keV)    (cps)    (cps) detection (s)
  0- 13  0.00- 6.14     0.474  7.04  835.848 ( 1862.872)
 14- 17  6.14- 7.90     0.023  2.31 1.12E+05 (*********)
 18- 23  7.90- 10.5  3.83E-03  3.04 5.19E+06 (*********)
 24- 35  10.5- 15.8  2.62E-04  5.24 1.91E+09 (*********)
 36- 49  15.8- 22.1  1.20E-06  5.93 1.02E+14 (*********)
 50-249  22.1-116.0  2.12E-09 37.37 2.07E+20 (*********)
PIMMS > mo plasma 1.0 1.0 1e20
NOTE: This version of PIMMS has a grid of  59x 5 grid of APEC models
      from kT= 0.034 keV (logT= 5.60) to kT=27.250 keV (logT= 8.50)
      and abundances from 0.20 to 1.00
         Selected temperature is  0.967 keV (log T is  7.05)
         and selected abundance is 1.0
PIMMS > plasma mekal
Current model has been changed
* Current model is PLASMA (mekal) with
                        kT= 0.9669keV (logT= 7.05), Abund=1.0; NH =  1.000E+20
PIMMS > go 1
* For PLASMA (mekal) model with
                        kT= 0.9669keV (logT= 7.05), Abund=1.0; NH =  1.000E+20
  and  1.000E+00 cps in EXOSAT ME
  (Internal model normalization =  3.312E-02)
* PIMMS predicts  2.531E-01 cps with XTE PCA
   (Count rate is per PCU)

%%%        With 3 PCUs operational:
           (Use these numbers in RPS)

PIMMS predicts     0.759 cps from the source plus    91.380 background cps
5-sigma detection will be achieved in 3996.702s
(but undetectable with 1% systematic uncertainties in bgd)

Results in the 6 canonical XTE PCA bands are:

Channels  Nominal    Source   BGD   5-sigma    (+1%)
          E (keV)    (cps)    (cps) detection (s)
  0- 13  0.00- 6.14     0.722 10.56  540.761 ( 1161.651)
 14- 17  6.14- 7.90     0.032  3.46 8.63E+04 (*********)
 18- 23  7.90- 10.5  4.88E-03  4.56 4.80E+06 (*********)
 24- 35  10.5- 15.8  2.76E-04  7.87 2.58E+09 (*********)
 36- 49  15.8- 22.1  8.03E-07  8.90 3.45E+14 (*********)
 50-249  22.1-116.0  7.73E-10 56.05 2.35E+21 (*********)

%%% ...and with 2 PCUs operational:

PIMMS predicts     0.506 cps from the source plus    60.920 background cps
5-sigma detection will be achieved in 6.00E+03s
(but undetectable with 1% systematic uncertainties in bgd)

Results in the 6 canonical XTE PCA bands are:

Channels  Nominal    Source   BGD   5-sigma    (+1%)
          E (keV)    (cps)    (cps) detection (s)
  0- 13  0.00- 6.14     0.481  7.04  811.142 ( 1742.477)
 14- 17  6.14- 7.90     0.021  2.31 1.29E+05 (*********)
 18- 23  7.90- 10.5  3.25E-03  3.04 7.20E+06 (*********)
 24- 35  10.5- 15.8  1.84E-04  5.24 3.87E+09 (*********)
 36- 49  15.8- 22.1  5.35E-07  5.93 5.18E+14 (*********)
 50-249  22.1-116.0  5.15E-10 37.37 3.52E+21 (*********)
PIMMS > quit
\end{verbatim}

In this example, the user specified conversion from EXOSAT ME count
rate to XTE PCA and used plasma models with two different absorbing
columns, also switching from APEC (default) to mekal before the second
run.  Since PIMMS uses a grid of pre-calculated plasma models, the temperature
of the actual model used does not exactly match the request.
PIMMS provides instrument-specific information for the
XTE PCA (source and background count rates, and 5$\sigma$ detection times
in the entire passband and in the 6 canonical PCA bands with 2 or 3 PCUs
operational).

\subsection*{Example 3. Estimating XTE count rates II}

\begin{verbatim}
*** PIMMS version 4.0 ***
    2010 Feb 17th Release
    Reading mission directory, please wait
* Current model is BREMSSTRAHLUNG, kT=  10.0000 keV; NH =  1.000E+21
   <--- Use 'MODEL' command to change
        and 'PLASMA' command to switch among APEC/mekal/RS
* By default, input rate is taken to be
 Flux (    2.000-   10.000 keV) in ergs/cm/cm/s
   <--- Use 'FROM' command to change the default
* Simulation product will be
 Count rate in CHANDRA ACIS-I
   <--- Use 'INSTRUMENT' command to switch to another instrument
PIMMS > from ginga lac both
PIMMS > mo pl 1.5 15 30 1e22
PIMMS > inst xte pca
PIMMS > go 500
* For power-law model with high-energy cut-off with
  Index =  1.50, Ecut   15.00 keV, E(e-folding)   30.00 keV; NH =  1.000E+22
  and  5.000E+02 cps in GINGA LAC BOTH
  (Internal model normalization =  2.001E-01)
* PIMMS predicts  1.456E+02 cps with XTE PCA
   (Count rate is per PCU)

%%%        With 3 PCUs operational:
           (Use these numbers in RPS)

PIMMS predicts   436.940 cps from the source plus    91.380 background cps
5-sigma detection will be achieved in    0.069s
(or in    0.069s with 1% systematic uncertainties in bgd)

Results in the 6 canonical XTE PCA bands are:

Channels  Nominal    Source   BGD   5-sigma    (+1%)
          E (keV)    (cps)    (cps) detection (s)
  0- 13  0.00- 6.14   160.342 10.56    0.166 (    0.166)
 14- 17  6.14- 7.90    66.161  3.46    0.398 (    0.398)
 18- 23  7.90- 10.5    75.825  4.56    0.350 (    0.350)
 24- 35  10.5- 15.8    84.634  7.87    0.323 (    0.323)
 36- 49  15.8- 22.1    34.018  8.90    0.927 (    0.927)
 50-249  22.1-116.0    15.961 56.05    7.067 (    7.292)

%%% ...and with 2 PCUs operational:

PIMMS predicts   291.293 cps from the source plus    60.920 background cps
5-sigma detection will be achieved in    0.104s
(or in    0.104s with 1% systematic uncertainties in bgd)

Results in the 6 canonical XTE PCA bands are:

Channels  Nominal    Source   BGD   5-sigma    (+1%)
          E (keV)    (cps)    (cps) detection (s)
  0- 13  0.00- 6.14   106.895  7.04    0.249 (    0.249)
 14- 17  6.14- 7.90    44.107  2.31    0.596 (    0.596)
 18- 23  7.90- 10.5    50.550  3.04    0.524 (    0.524)
 24- 35  10.5- 15.8    56.423  5.24    0.484 (    0.484)
 36- 49  15.8- 22.1    22.678  5.93    1.391 (    1.391)
 50-249  22.1-116.0    10.640 37.37   10.601 (   10.938)
PIMMS > inst xte hexte def
PIMMS > go 500
* For power-law model with high-energy cut-off with
  Index =  1.50, Ecut   15.00 keV, E(e-folding)   30.00 keV; NH =  1.000E+22
  and  5.000E+02 cps in GINGA LAC BOTH
  (Internal model normalization =  2.001E-01)
* PIMMS predicts  1.834E+01 cps with XTE HEXTE DEFAULT
   (Source-only count rate in 1 cluster; BGD rate is 73.1 per cluster)
5-sigma detection will be achieved in     14.0s

Results in the 4 canonical XTE HEXTE bands are:
             (per HEXTE cluster)

  Channels Nominal   Source   BGD   5-sigma
           E (keV)   (cps)    (cps) detection (s)
    5- 29   12-  30    13.9   11.86      5.4
   30- 61   30-  62     3.7   17.93     82.3
   62-125   62- 126  6.73E-01 21.94   2855.4
  126-250  126- 250  1.38E-02 21.35 6.53E+06
 (The default 16-s rocking cycle is assumed for detection time)
PIMMS > quit
\end{verbatim}

In this case, a special version of the power law model (with high
energy cut-off) is used, by specifying index, cut-off energy and
e-folding energy, as well as Nh, on the command line.  User then
calculated PCA and HEXTE count rates for a 500 cps Ginga LAC source.

\subsection*{Example 4. Estimating ROSAT PSPC count rates}

\begin{verbatim}
*** PIMMS version 4.0 ***
    2010 Feb 17th Release
    Reading mission directory, please wait
* Current model is BREMSSTRAHLUNG, kT=  10.0000 keV; NH =  1.000E+21
   <--- Use 'MODEL' command to change
        and 'PLASMA' command to switch among APEC/mekal/RS
* By default, input rate is taken to be
 Flux (    2.000-   10.000 keV) in ergs/cm/cm/s
   <--- Use 'FROM' command to change the default
* Simulation product will be
 Count rate in CHANDRA ACIS-I
   <--- Use 'INSTRUMENT' command to switch to another instrument
PIMMS > from flux ergs 0.1-4
PIMMS > mo plasma 6.65 logt 0.6 3e19
PIMMS > inst rosat pspc open
PIMMS > go 5e-12
* For PLASMA (APEC) model with
                        kT= 0.3849keV (logT= 6.65), Abund=0.6; NH =  3.000E+19
   and a flux (    0.100-    4.000keV) of  5.000E-12 ergs/cm/cm/s
  (Internal model normalization =  3.179E-03)
* PIMMS predicts  7.113E-01 cps with ROSAT PSPC OPEN
PIMMS > inst rosat pspc r6r7
PIMMS > go 5e-12
* For PLASMA (APEC) model with
                        kT= 0.3849keV (logT= 6.65), Abund=0.6; NH =  3.000E+19
   and a flux (    0.100-    4.000keV) of  5.000E-12 ergs/cm/cm/s
  (Internal model normalization =  3.179E-03)
* PIMMS predicts  1.267E-01 cps with ROSAT PSPC R6R7
PIMMS > quit
\end{verbatim}

In this example, 0.6 Solar abundance, log$T$=6.65 APEC model with
an absorption of 3$\times10^{19}$ cm$^{-2}$ is used to estimate ROSAT PSPC
count rates (total and in a combination of the Snowden R bands) for a
5$\times 10^{-12}$ ergs\,cm$^{-2}$s$^{-1}$ source.

\section{Using multi-component models -- 3 examples}

\subsection{A 2-temperature plasma model}

\begin{verbatim}
*** PIMMS version 4.0 ***
    2010 Feb 17th Release
    Reading mission directory, please wait
* Current model is BREMSSTRAHLUNG, kT=  10.0000 keV; NH =  1.000E+21
   <--- Use 'MODEL' command to change
        and 'PLASMA' command to switch among APEC/mekal/RS
* By default, input rate is taken to be
 Flux (    2.000-   10.000 keV) in ergs/cm/cm/s
   <--- Use 'FROM' command to change the default
* Simulation product will be
 Count rate in CHANDRA ACIS-I
   <--- Use 'INSTRUMENT' command to switch to another instrument
PIMMS > mo plasma 6.6 logt 1.0 3e19 plasma 7.2 logt 1.0 3e19 0.5 1.0
PIMMS > output apec2t 0.1 4.0 0.002
PIMMS > inst rosat pspc open
PIMMS > from flux ergs 0.1-2.0 u
PIMMS > go 3e-11
* For PLASMA (APEC) model with
                        kT= 0.3431keV (logT= 6.60), Abund=1.0; NH =  3.000E+19
    + PLASMA (APEC) model with
                        kT= 1.3657keV (logT= 7.20), Abund=1.0; NH =  3.000E+19
             (  0.5000 times component 1 at     1.0000 keV)
   and an unabsorbed flux (    0.100-    2.000keV) of  3.000E-11 ergs/cm/cm/s
  (Internal model normalization =  1.180E-02)
* PIMMS predicts  3.846E+00 cps with ROSAT PSPC OPEN
PIMMS> quit
\end{verbatim}

This first example illustrates the use of two-temperature plasma
model.  The absorption columns (to be specified explicitly for each component)
are the same in this example.  The second component has a flux at 1 keV which
is 50\% of the first component.  Note, however, this is a tricky proposition
for the line-rich plasma models.  It is best to check this via the output
command, which allows the users to check the actual model spectrum.

\subsection{A model with partial-covering absoerber}

\begin{verbatim}
*** PIMMS version 4.0 ***
    2010 Feb 17th Release
    Reading mission directory, please wait
* Current model is BREMSSTRAHLUNG, kT=  10.0000 keV; NH =  1.000E+21
   <--- Use 'MODEL' command to change
        and 'PLASMA' command to switch among APEC/mekal/RS
* By default, input rate is taken to be
 Flux (    2.000-   10.000 keV) in ergs/cm/cm/s
   <--- Use 'FROM' command to change the default
* Simulation product will be
 Count rate in CHANDRA ACIS-I
   <--- Use 'INSTRUMENT' command to switch to another instrument
PIMMS > mo brems 15 3e23 brems 15 1e20 0.1 10 ga 6.5 0.1 250
PIMMS > output partial 0.1 10.0 0.005
PIMMS > inst xmm pn thin
PIMMS > go 0.5 asca sis
* For thermal Bremsstrahlung model with kT= 15.0000 keV; NH =  3.000E+23
    + thermal Bremsstrahlung model with kT= 15.0000 keV; NH =  1.000E+20
             (  0.1000 times component 1 at    10.0000 keV)
    + Gaussian model with E=  6.5000 keV; sigma= 0.1000 keV; NH =  3.000E+23
                          (Eq.W=250.0000 eV)
  and  5.000E-01 cps in ASCA SIS
%!% Integration over the entire chip (not just in the source region) assumed
  (Internal model normalization =  1.122E-02)
* PIMMS predicts  3.576E+00 cps with XMM PN THIN
  for on-axis observation before dead time correction,
  PATTERN=0 only rate assuming a 15 arcsec extraction radius
  (the total PATTERN=0 count rate is approx. 47% higher)
WARNING: The pile-up estimate is approximate

% Pile-up and dead-time corrected count rates in 4 energy bands
  using various window options are:

 Window    Pileup  Dead                Corrected Good Count Rates
   Option   frac.   Time    0.1-0.4   0.4-1.0   1.0-2.5  2.5-10.0     Total

 Full      0.939%   7.0%     0.4934    0.6910    0.6684    1.2785      3.2941
 FullExtd  4.495%   2.0%     0.5013    0.7020    0.6791    1.2989      3.3466
 Large     0.540%   9.0%     0.4848    0.6789    0.6567    1.2561      3.2363
 Small     0.017%  29.0%     0.3802    0.5324    0.5150    0.9852      2.5383
 Timing   -------   1.5%     0.5276    0.7388    0.7146    1.3670      3.5220
 Burst    -------  97.0%     0.0161    0.0225    0.0218    0.0416      0.1073

% Any pile-up predictions over 10% are highly uncertain

PIMMS> quit
\end{verbatim}

This example shows how to set up a spectrum suffering from partial
covering absorption.  In addition, a Gaussian is added which has
a different syntax: the three parameters are the line energy (keV),
the physical width (keV), and the equivalent width (eV).  Absorbing
column is assumed to be the same as the first component (it can
be specified explicitly, as the third parameter before the equivalent
width).

\subsection{A redshifted example}

\begin{verbatim}
*** PIMMS version 4.0 ***
    2010 Feb 17th Release
    Reading mission directory, please wait
* Current model is BREMSSTRAHLUNG, kT=  10.0000 keV; NH =  1.000E+21
   <--- Use 'MODEL' command to change
        and 'PLASMA' command to switch among APEC/mekal/RS
* By default, input rate is taken to be
 Flux (    2.000-   10.000 keV) in ergs/cm/cm/s
   <--- Use 'FROM' command to change the default
* Simulation product will be
 Count rate in CHANDRA ACIS-I
   <--- Use 'INSTRUMENT' command to switch to another instrument
PIMMS > mo pl 1.7 3e23 ga 6.4 0.2 150 plasma 1.2 1.0 0.0 1.0 2.5 z 0.01 8e19
NOTE: This version of PIMMS has a grid of  59x 5 grid of APEC models
      from kT= 0.034 keV (logT= 5.60) to kT=27.250 keV (logT= 8.50)
      and abundances from 0.20 to 1.00
         Selected temperature is  1.217 keV (log T is  7.15)
         and selected abundance is 1.0
PIMMS > output agn_sb 0.1 10.0 0.01
PIMMS > inst chandra acis-s
PIMMS > go 1e-11
* For power law model with photon index = 1.7000; NH =  3.000E+23
    + Gaussian model with E=  6.4000 keV; sigma= 0.2000 keV; NH =  3.000E+23
                          (Eq.W=150.0000 eV)
    + PLASMA (APEC) model with
                        kT= 1.2172keV (logT= 7.15), Abund=1.0; NH =  0.000E+00
             (  1.0000 times component 1 at     2.5000 keV)
 ...redshifted with z=  0.0100 and a Galactic Nh= 8.000E+19
   and a flux (    2.000-   10.000keV) of  1.000E-11 ergs/cm/cm/s
  (Internal model normalization =  7.355E-03)
* PIMMS predicts  1.711E-01 cps with CHANDRA ACIS-S
% Pileup estimate for ACIS:
  Pile-up is tolerable (10.0 %) at a frame-time of  1.546 s
PIMMS > quit
\end{verbatim}

In this example, the entire composite model is to be redshifted with z=0.01.
Absorption specified with each component is taken to be intrinsic (i.e.,
also redshifted).  Additionally, a Galactic (unredshifted) absorption is
specified.  For `unabsorbed' flux, both intrinsic and Galactic absorption
will temporarily be set to 0.

\section{Comments on extended sources}

PIMMS is written primarily for point sources.  To simulate the count
rate for an extended source, estimate the total counts/flux within the
field of view of the target instrument and use that as the input to
the ``GO'' command.

\section{Missions}

PIMMS reads the list of missions from a file called ``pms\_mssn.lst'' in the
data directory.     It then looks, for each mission (i.e., satellite),
detector and ``filter'' combination, the  appropriate calibration files for the
effective area etc.   Since this is a run-time process, the following items
may not exactly correspond to what you see.   For a listing of what is
currently available, use the DIRECTORY command.

For active and near-future missions, we provide the latest effective area
curves with PIMMS for proposal preparation purposes.  If the effective area
changes in-orbit, count rate to flux conversion factor for actual observations
is time-dependent, which PIMMS is currently not well equipped to handle.

\subsection{ASCA}

The Japanese X-ray satellite {\em ASCA\/} had 4 co-aligned telescopes,
each having an effective area  of $\sim$250 cm$^2$ at 1 keV; there were
two GIS (imaging gas scintillation proportional counters) and two SIS
(Solid-state Imaging Spectrometer, X-ray CCDs) detectors.  Count rates
are given for a single GIS or a single SIS, as appropriate.

\subsection{ASTROH}

{\em ASTRO-H\/} is the next Japanese X-ray astronomy satellite with a
substantial US contribution, scheduled for launch in early 2016.  PIMMS
v4.8 contains effective area curves for the 4 instruments based on
preliminary calibration available as of 2015 July.

Soft X-ray Spectrometer (SXS) is an X-ray microcalorimeter based instrument
located at the focus of an imaging soft X-ray telescope (SXT-S).  We provide
four files corresponding to the current set of science filters (open, 25
micron Be, and 50 micron Be, as well as the optical blocking filter, OBF).
These files are appropriate for point sources on-axis, adding counts in all
pixels of the 6x6 array.  For the detector, a spectral resoulution of 5 eV
and the baseline set of XRS filters are assumed.  For bright sources, SXS
suffers from a photon pile-up effect: the fraction of events that can be
detected at the full resolution will decrease for higher count rate source.
The current version provides a preliminary estimate of this effect, the
accuracy of which is under investigation.

Soft X-ray Imager (SXI) is an CCD based instrument located at the focus of
a soft X-ray telescope (SXT-I).  The effective area curve is appropriate for
a point source observed on-axis, analyzed with a 1.8 arcmin radius extraction
region.

Hard X-ray Imager (HXI) is a CdTe imaging detector behind 4 Si layers.
There are two units, each located at the focus of an imaging, multi-layer,
hard X-ray telescope (HXT).  The effective area curves are for one HXI
unit for an on-axis point source with 1.8 arcmin extraction region.
Two choices are offered for top layer only and and for all layers.

Soft Gamma-ray Detector (SGD) is a narrow field-of-view Compton telescope
operating in the 10-600 keV range.  Its sensitivity at 300 keV is 10 times
better than that of the Suzaku HXD.  The current version provides preliminary
estimates of the exposure times necessary for 3 and 5 $\sigma$ detections with
SGD, the accuracy of which is under investigation.

\subsection{BBXRT}

BBXRT is flown on the Shuttle with the ASTRO payload in December 1990.   The
effective area curve  is that for pixel A0.

\subsection{CGRO}

The Compton Gamma Ray Observatory OSSE instrument is now available as
in PIMMS, primarily as an aid in Integral observation planning.

\subsection{Chandra}

{\it The Chandra X-ray Observatory was formerly known as AXAF; starting
with v2.7, the mission name in PIMMS has been changed to Chandra.}

This version of PIMMS includes the Chandra instrument effective area curves
appropriate for Cycle 18 proposers, as provided by the Chandra X-ray
Observatory Center (http://chandra.harvard.edu/), where the details of the
instruments can be found.  Older versions of the effective area curves are
available by request.

All effective areas assume an on-axis observation.

The 4 CCDs of the Chandra CCD Imaging Spectrometer Imaging array (ACIS-I)
cover ~17 by 17 arcmin of sky with ~0.5 arcsec square pixels.  PIMMS
calculates the on-axis count rate uncorrected for pile-up.

Two of the 6 ACIS-S CCDs are back-illuminated (BI), to improve the low
energy effective areas. For imaging with ACIS-S, the observer will thus
have the choice of FI/BI.  Since the FI chips have a performance identical
to that of the ACIS-I chips, only the BI chip option is separately available
in PIMMS.

High Resolution Camera covers a larger area of the sky with smaller pixels.

ACIS-S will be the normal readout instrument for HETG
(High Energy Transmission Grating) spectra. The curves
use the flight instrument chip layout of 4 FI and 2 BI chips. PIMMS will
provide count rates for both grating subsets (MEG and HEG) separately, or
for the combination. A single observation provides both spectra
simultaneously in a cross-shaped orientation. Since the energy response and
spectral resolution of the two grating assemblies differ, separation of the
output may be important for some users. In all cases the output is for
isolated first order. The energy resolution of the ACIS instrument will
allow the user to separate this from the higher order light. See the Proposers'
Observatory Guide for more information.  Count rate in the zeroth order
image can also be calculated.

HRC-S is the normal readout instrument for LETG (Low Energy
Transmission Grating) spectra. PIMMS currently
allows determination of the count rates in the 0th order image; 1st order
spectrum; and the higher orders; through the "normal" part of the UV/Ion
shield.  NOTE in practice, due to the lack of energy resolution of the HRC,
isolation of the first order signal will require a combination of "normal"
and "low-energy reject" mode observations which are likely to take roughly
twice as long as the estimate provided by PIMMS. An alternative is to use
the High Energy Suppression Filter (HESF) which effectively isolates first
order from 0.05-0.44 keV.

LETG data can also be read out with the ACIS-S detectors; 0th and 1st order
count rates for this combination can also be estimated with PIMMS.

Observations of bright sources with ACIS are limited by photon pileups
(see Proposers' Observatory Guide).  This version of PIMMS includes a
pileup estimate (based on the separate `pileup' tool written at CXC).
This feature will provide you with an estimate of the degree of pile-up
for ACIS imaging mode observations (ACIS-I, ACIS-S, LETG-ACIS-I ORDER0,
LETG-ACIS-S ORDER0, HETG-ACIS-I ORDER0, and HETG-ACIS-S ORDER0).

Note that this is valid only for point source observations on-axis.

Pile-up effect can be mitigated by placing the source off-axis --- the
inferior PSF will spread the photons over many pixels.  Quantitative
analysis of this is not yet available in PIMMS.   The other principal
method of altering the frame time can be evaluated by PIMMS.  For this
purpose, the 'go' command of PIMMS for ACIS-I and ACIS-S-BI allows an
optional numerical parameter.  If given, it will be taken as the frame
time (allowed range: 0.2--3.2 s), and gives the pile-up fraction
accordingly.  If absent, PIMMS will attempt to estimate the frame time
at which the pile-up fraction is 10\% (which is the rule-of-thumb number
above which you will have a severe problem).  For the 0th order
images, the default frame time of 3.2 s is assumed.

\subsection{EINSTEIN}

Currently PIMMS only have IPC and MPC effective area curves.

\subsection{EUVE}

PIMMS currently has the effective area curves for the three channels of  the
spectrometer, which is used by GOs for pointed observations.  Detectors
are SW (70--190\AA), MW (140--480\AA) and LW (280--750\AA)

\subsection{EXOSAT}

For the Low Energy telescopes, only the LE1/CMA effective area data
are kept within PIMMS. Specify filter OPEN, LX3, LX4, ALP, or BRN.
The ME effective area is for a half-array; GSPC area is also available.

\subsection{Ginga}

Ginga is the 3rd Japanese X-ray astronomy satellite, which carried the LAC
(Large Area Counter) array with an effective area of $\sim$4000 cm$^2$.
Count rate can be calculated for TOP layer of the detector only or BOTH.

\subsection{HEAO-1}

PIMMS currently contains supports for A1 and A4 LED instruments.

The PIMMS set-up for the A4 LED instrument is meant to make it straightforward
to use the Levine et al (1984, ApJS 54, 281) catalog for XTE proposals.  As an
input, use the A+B+C combined count rate; as an output, A+B+C rate as well
as the individual rate in the four bands (A through D) are given.  One
suggested use is to specify ``HEAO1 A4'' as both input and output instrument:
by an iterative process, the user can find a spectral model that reproduces
the distribution of counts in different bands.  Then switch to a different
output instrument (in terms of energy range, LED matches the higher end of
XTE PCA and the lower end of XTE HEXTE) keeping that model.

For the A1 instruments, effective area curves for two gain settings, AGCL
and AGCP. Both are normalized to produce count rates per square cm to allow
direct comparison with HEAO-1 A1 X-ray Source Catalog (Wood et al. 1984,
ApJSupp, 56, 507; also available online). The AGCP version is applicable to
the majority of sources, while the AGCL setting is appropriate for sources at
ecliptic longitudes in the 230-265 and 50-85 deg ranges (but excluding those
at ecliptic latitude $>+$80 or $<-$80). The effective area curves were
digitized from Fig. 7 of Wood et al. by Koji Mukai, while Dr. Kent Wood of
NRL kindly provided additional information.

\subsection{Integral}

As of version 3.9f, PIMMS includes effegtive area curves for ISGRI and
JEM-X instruments, based on calibration as of 2008 March.
See http://obswww.unige.ch/isdc/ for details of the mission.
Since 2010 October, two units of JEM-X are operational, but PIMMS continues
to provide count rate per one unit. This is reflected in the mission-specific
comments since PIMMS v4.3.

\subsection{IXO}

The International X-ray Observatory (IXO) was a proposed joint effort of NASA,
ESA, and JAXA, and superceded the previously planned Constellation-X and XEUS
missions.  (Note that the support of Con-X and XEUS missions has been
discontinued in PIMMS.)  PIMMS v4.1b contains effective area curves
obtained from the IXO project as of 2010 August.

\subsection{MAXI}
MAXI (Monitor of All-sky X-ray Image) is an all-sky monitor on-board the
International Space Station. MAXI payload includes two types of X-ray cameras,
the Gas Slit Camera (GSC) and the Solid-state Slit Camera (SSC). PIMMS v4.7
contains effective area curve files provided by the MAXI team in mid-2013,
normalized to produce count rates per square cm.  More information on MAXI
can be obtaeind from the MAXI home page at Riken (http://maxi.riken.jp/top/).

\subsection{NICER}

NICER (the Neutron star Interior Composition ExploreR) is a proposed X-ray
astrophysics payload on the International Space Station. It builds up total
effective area in a modular way, with 56 parallel optic-detector pairs. Each
optic is an X-ray ``concentrator''-- a non-imaging grazing-incidence mirror
assembly that focuses a 6-arcmin-diameter patch of sky onto a detector at a
focal distance of 108.5 cm. The detectors are of the silicon-drift variety,
with CCD-like energy resolution. Background levels are anticipated to be
approximately 0.1 cts/sec (at high Galactic latitudes) from the diffuse
X-ray background, most of which will be below 2 keV, and 0.1 cts/sec/keV
from unrejected particle detections masquerading as X-rays.

The current effective-area file was provided by the NICER project in 2014
December. Note that the NICER project is in the process of making design
choices that may affect the effective area curves, so the current version
should be taken as preliminary.

\subsection{NuSTAR}

NuSTAR is an X-ray satellite with 10-meter focusing length and two
side-by-side focal plane modules, each with 4 detectors.  The count rate
calculated by PIMMS reflects the total from both modules for a 50\% PSF
extraction region. In Version 4.7, the NuSTAR effective area curve file
and related mission specific information were updated to reflect post-launch
understanding, based on the response and background files for simulation
released in 2013 August.For further details about this mission, see
  http://www.nustar.caltech.edu/

\subsection{ROSAT}

For the German XRT, effective area curve with PSPC (filter OPEN or BRN)
and HRI are available.  Also, beginning with v2.3, the Snowden R bands
(see Snowden et al 1994, ApJ 424, 714) are available as software filters
(R1, R1R2, R4, R4R5, R4TOR7, R5, R6, R6R7, and R7).
For the British WFC, filters S1, S2, P1 and P2
effective area curves are available; these are
appropriate for the time of launch. Note the S1 and S2 sensitivity dropped
to $\sim$75\% of initial value by the end  of the survey,  followed by a steeper
decline to 15--20\% of the original value after The Tumble.  Non-survey (P1
and P2) filters have suffered much smaller degradation.

\subsection{SAX}

Although the official name for this Italian-Dutch satellite is now
BeppoSAX, the mission name within PIMMS remains ``SAX''.  It
was launched in Apr 1996 by an Atlas G-Centaur directly into
a 600 km orbit at 3 degrees inclination.
SAX carries 4 narrow field instruments (1 LECS, 2 MECS, 1 HPGSPC, 1 PDS),
covering the energy band from 0.1 to 200 keV, and two Wide Field Cameras
(WFC, 2-30 keV) which view the sky through a coded mask perpendicularly
to the axis of the narrow field instruments.
The LECS (0.1-10 keV) is an imaging gas scintillation proportional
counter similar to the MECS but extends the energy range down to 0.1 keV.
The MECS (1-10 keV) is an imaging gas scintillation proportional counter
similar to the LECS. There are 2 working MECS on board SAX (a third unit
developed a fault in May 1997). The count rate estimate is for the 2 MECS,
starting with version 2.4b (previous versions estimated for 3 MECS).
The HPGSPC is an high pressure gas scintillation proportional
counter sensitive in the energy range 3-120 keV with a FOV of 1 deg.
The PSD, phoswich detector system, consists in four phoswich units.
The observations are carried out with two halves of the
experiment alternatively pointing source and background region, providing
a continuous monitoring of the background.
The PSD is sensitive in the 15-300 keV energy bandwidth and has a 
FOV of 1.5 deg.
The Wide field Cameras is position sensitive proportional counter
sensitive in the 2-30 keV band. There are 2 WFC on board SAX. The FOV
per unit is 20 deg X 20 deg with an angular resolution of a few arcmin.

\subsection{Suzaku}

{\em Suzaku\/} (formerly {\em Astro-E2\/}) is a Japanese-US X-ray astronomy
satellite, launched in July 2005.  The current PIMMS implimentation is based
on information from the instrument teams as of 2012 September for the HXD
and 2014 December for the XIS, as collected by the Suzaku GOF
(http://suzaku.gsfc.nasa.gov/).

Note, however, that AO-10 proposals must not require the HXD to
achieve its core scientific objectives, as the satellite is operated
without the HXD for the majority of time.

The Hard-Xray Detector (HXD) is a non-imaging instrument with an effective
area of $\sim$300 cm$^2$ featuring a compound-eye configuration and an
extremely low background.  It consists of two types of sensors, silicon
PIN diodes and GSO crystal scintillators.

Starting with v3.9h, PIMMS also outputs approximate time needed for
3-sigma and 5-sigma detections in standard energy bands (1 for PIN
and 2 for GSO), taking into account the current level of systematic
uncertainties in the background models.

There are four units of the X-ray Imaging Spectrometer (XIS)
on-board {\em Suzaku,\/}
three with frontside-illuminated (FI) CCDs and one with a backside-illuminated
(BI) CCD, although XIS-2 (with an FI chip) has become inoperative
in November, 2006. Each XIS detector is located at the focus of a conical foil
X-Ray Telescope (XRT) with a 4.75m focal length.  The CCD pixels of XIS
vastly oversamples the XRT PSF, thereby allowing high S/N spectroscopy
with a relatively benign amount of photon pile-up.

PIMMS currently returns count rate per one unit of XIS, with no further
instrument specific information.

The X-Ray Spectrometer (XRS) lost its liquid helium cryogen and
is no longer operational.  A pre-launch estimate of the XRS effective
area is included for reference.

\subsection{Swift}

Swift (see {\tt http://heasarc.gsfc.nasa.gov/docs/swift/swiftsc.html}) is
a multiwavlength gamma-ray burst observatory launched on 2004 November 20.
Swift carries a wide-field (2 sr), coded-aperture Burst Alert Telescope
(BAT, 15-150 keV); an X-Ray Telescope (XRT, 0.2-10 keV); and a UV/Optical
Telescope (UVOT, 170-650 nm).  The effective area curves for XRT now
reflects the 2008 July CALDB release, for photon counting and window timing
modes.  The photodiode mode (no longer operative) is also included
using an older calibration.  Of the UVOT (9 filters, including ``white''
for unfiltered and 2 grisms) effective areas, those for the 2 grisms
are pre-launch estimates, and others (in PIMMS 3.9f) are derived from
the 2007 May CALDB release.  The BAT response in PIMMS v3.6c and later
yields the counts per fully illuminated detector, which matches the BAT
analysis software default units.  One detector has a geometric area of
0.16 cm$^2$.  An on-axis source illuminates 16384 detectors; PIMMS v3.6b
and earlier calculated the total on-axis count rates (i.e., per 16384
detectors).

Note that PIMMS is primarily an X-ray tool, and extrapolation to the
UV regime introduces additional uncertainties.  In particular, PIMMS
assumes E$_{\rm B-V}$ = N$_{\rm H}$ / 4.8$times 10^{21}$
and an average Milkyway extinction law.

\subsection{XMM}

XMM-Newton (implemented using the pre-launch name, XMM, in PIMMS) was launched
successfully in 1999 December.  It consists of three coaligned
high-throughput 7.5m focal length telescopes with six arc second (FWHM)
angular resolution.  The European Photon Imaging Camera (EPIC), which
consists of two MOS and one PN CCD arrays, provide moderate spectral
resolution over a30 arc minute field of view.  High-resolution
spectral information (E/dE$\sim$300) is provided by the Reflection Grating
Spectrometer (RGS) that deflects half of the beam on two of the X-ray
telescopes (those with the MOS arrays).  The observatory also has a
coaligned 30cm optical/UV telescope called the Optical Monitor (OM).  
See http://astro.estec.esa.nl/XMM/xmm.html for further details.

The count rates for the EPIC MOS are given for one instrument each
(we have averaged of MOS1 and MOS2 effective area curves), not for
pairs of instruments.  Starting with PIMMS v4.1, the EPIC count rates are
given for PATTERN=0--12 (MOS)/4 (pn) events over a large (5 arcmin)
extraction region.  This reverses the practice adoptied in v3.6
to use a 15 arcse radius extraction region, and that adopted in
v3.9c to report only the PATTERN=0 count rates for PN.

For the RGS, count rates in RGS1 and RGS2 in two orders can be calculated
separately (i.e., a total of 4 possible combinations).
Even though the term ``filter'' is used (because that's what the most
common use of the third level of instrument specification in PIMMS),
these do not represent physical filters.  Data are taken in all three
orders simultaneously, to be extracted into separate spectra using
software filters.

Current version of PIMMS contains effective area curves appropriate for
AO-15 proposals.

\subsection{XTE}

XTE, which was launched in Dec 1995,
carries the All-Sky Monitor (ASM), large area proportional counter array
(PCA) and the high energy X-ray Timing Experiment (HEXTE).

PCA is a mechanically-collimated array of five xenon proportional counter
units (PCUs) with a total effective area of $\sim$7000 cm$^2$; however,
different observations are taken with difeerent numbers of PCUs on.  Therefore,
starting with PIMMS v2.7, user must supply the count rate {\sl per PCU\/}
when this is used as the input mission (`from xte pca').  When used as the
output mission (`inst xte pca'), the first output is count rate per PCU summed
over all energies and over all 3 xenon layers.  Additional outputs (the rates
in the 6 canonical PCA channels required on the proposal form and used in
RECOMMD) are given for 3 and 2 PCU combinations, which are becoming more
frequent (and the proposal form requires numbers for 3 PCUs).  The effective
area curves, channel boundaries and the extimated background rates are all
appropriate for `Epoch 4' gain setting.

HEXTE consists of two clusters of
detectors, with 4 scintillation detectors in each cluster.  Count rates are 
given per cluster.  Values are given for the total count rate, and the count 
rates in the 4 canonical HEXTE channels required on the proposal form and 
used in HEXTEmporize.

The quoted detection times assume two-cluster 16-s source/background
beamswitching, i.e., one cluster measures background while the other is
on-source.  In this case, the ``detection time'' applies to the combined
HEXTE instrument.  For those bright source observations (source rate $>>$
background rate) where a HEXTE cluster is selected to be in STARE mode,
this detection time can also be interpreted as appropriate for a single
HEXTE cluster.  For the combined HEXTE detection time, divide by $\sqrt{2}$.

\subsection{Flux}

PIMMS can also calculate conversion to/from flux values not folded through
any instrument responses can also be used.  To use flux, the unit must be
specified: ``ERGS'' for ergs\,cm$^{-2}$\,s$^{-1}$ or ``PHOTONS'' for
photons\,cm$^{-2}$\,s$^{-1}$.  Also necessary is the energy
range of interest, to be specified in the form ``2.5--10'' (for 2.5 to 10 keV;
PIMMS now accepts ``2.0-20 A'' to mean 2 to 20A range in wavelength).
Optional keyword ``UNABSORBED'' following the range will make PIMMS calculate
flux with Nh set to 0.0; this is useful in relating the flux to the total
bolometric luminosity of the X-ray source before interstellar absorption.

\subsection{(Flux) Density}

New in PIMMS v3.3: it can now convert to/from flux density at a specific
energy, rather than integrated flux over a range of energies (``flux'').
To use density, the unit must be specified: ``ERGS'' for
ergs\,cm$^{-2}$s$^{-1}$kev$^{-1}$ or ``PHOTONS'' for
photons\,cm$^{-2}$s$^{-1}$kev$^{-1}$.  Also necessary is the energy of
interest (in keV).  Alternatively, this can be specified as the wavelength
in Angstroms, with the optional argument ``Angstrom,'' in which case flux
density is in (photons or ergs)\,cm$^{-2}$s$^{-1}\AA^{-1}$.
Flux density can be ``unabsorbed'' as is flux.

\subsection{Normalization}

PIMMS also supports the use of 'model normalization'.  This is most useful
for modelx imported from xspec, in which case a normalization of 1.0 is
the flux level as simulated within xspec.  The use of normalization is not
recommended for models bulit-in to PIMMS, as they do not necessarily relate
to any physical quantities.

\section{User interface}

\subsection{Command interpreter}

Commands  can  be  abbreviated  (as  long  as it  is  unique),  numerical  and
character  string  parameters can  be passed onto  the commands.    Parameters
are  interpreted  according   to  their  positions  within  the  command  line 
(first string is input file name,  second file name is  output file name etc.,
although this does  not happen in PIMMS).   Some parameters  are compulsory 
--- PIMMS  will prompt  you  for  them if they are not given on the 
command line;
others are optional  (default values will  be used unless  the user  specifies
them  on the  command  line).   Commands can  be stringed  together by using a
semicolon.

PIMMS can switch command input to a file, using \@$<$filname$>$ convention.
PIMMS will assume .xco extension, if none is given.  Nested indirections
are not allowed.

\subsection{On-line help}

PIMMS contains a VMS-style help facility, within which
information is  stored in a hierarchical structure.   On the top level,
there are two types of topics.   Those listed in ALL CAPITAL LETTERS are PIMMS
command names,  containing the usage  of these commands.    Others are of more
general nature, not linked with specific commands.  HELP items are arranged in
many levels, so that you start with a general introduction and pick up as many
specific details as you like by going down several levels.

This help is case-insensitive  (i.e., it accepts both lower-case and capital
letters).   If you type n characters,  it will be matched against  the first
n characters of  the topic names at that level.   No wild card is allowed in
specifying item names.

If the topic name
string supplied by the user  can be matched to (parts of) two more more
topic names, then information on all the matching topics will be displayed.

Type ``?'' to repeat the current level.

Type $<$RETURN$>$ as the topic name to go up one level.

To exit HELP, type the EOF character (\^Z on VAX/VMS, \^D on UNIX machines).

\subsection{Spawn}

Enter a dollar sign ``\$'' followed by a command to be spawned.  Note that some
operating system may not pass aliases, environment variables, logicals etc.

\pagebreak

\section*{Appendix A. PIMMS commands}

\subsection*{MODEL}

Command syntax: MODEL $<$name$>$ $<$par$>$ $<$nh$>$ [$<$name$>$ $<$par$>$ $<$nh$>$ $<$ratio$>$ $<$refe$>$...] \\
\hspace{1.5 cm} or MODEL PL $<$par1$>$ $<$par2$>$ $<$par3$>$ $<$nh$>$ \\
\hspace{1.5 cm} or MODEL $<$filename$>$ [$<$nh$>$] \\
\hspace{1.5 cm} or MODEL ? \\
Minimum abbreviation: M \\
Examples: ``MO PL 1.7 3e21'' ``MO mymodel.dat'' \\
\vspace{0.5 cm}

Model specifies the spectral shape to be folded  with the effective area curve
of the instrument.  Starting with version 3.0, up to 8 model components can be
added together to represent multi-temperature plasma,  power law plus Gaussian
emission line, partial-covering absorber etc.   Only a limited combinations of
models have been rigorously tested;   users of complicated models are urged to 
check the composite model by using the OUTPUT command.

As components, PIMMS recognizes  BLACKBODY, POWERLAW, BREMSSTRAHLUNG, GAUSSIAN
and PLASMA.   If the model name string does  not match these, PIMMS will
try to interpret the string  as a file name containing  a precalculated  model
containing energy, photon flux pairs (see EXTERNAL\_MODELS; MODELS\_DIRECTORY).
Nh, the equivalent neutral hydrogen column density  (using Morrison \& McCammon
model) is expected for each component, except for GAUSSIAN and external files.
Nh should be specified in full with an appropriate exponent (e.g., 2.5e21), or
as a small non-zero number less than 30.0, to be intepreted as log10(Nh).

Model normalization for the built-in models do not necessarily correspond to
physical quantities.  Instead, PIMMS is designed to allow users to compare 2
observable quantities (count rates through specific instruments, or fluxes in
specific energy bands) without having to know the model normalization.

Optionally,  all components may be redshifted using a common z (in which case,
all component Nh are interpreted as intrinsic absorber, with the same z)  with
an optional Galactic NH.

\subsubsection*{Syntax}

The Model commands takes 1--8 blocks of component specification, and a final
optional block of redshift specification.  Each block starts with a valid
name of a component (including a valid file name), followed by a set of
numerical parameters.  As a special case, the plasma model takes an
optional string (`logt' or `kev', the latter being the default) specifying
the unit of temperature, immediately after the numerical parameter for the
temperature.  For 2nd through 8th
component blocks, additional parameter(s) specifying the ratio of that
component to component 1, and the reference energy at which this is to
be evaluated, must also be given (NB the reference energy for Gaussian
is always its central energy).

\subsubsection*{MODEL --- ?}

MODEL command  with a question mark  (or no  parameters) will return a short
listing of available models.

\subsubsection*{MODEL --- Blackbody}

Can be abbreviated to ``BL..'' or ``BB''.  Parameter is temperature in keV.

\subsubsection*{MODEL --- Bremsstrahlung}

Can be abbreviated to ``BR..'' or ``TB'' (short for Thermal Bremsstrahlung); the
model include the Gaunt factor.  Parameter is temperature in keV.

\subsubsection*{MODEL --- Power Law}

Can be  abbreviated to ``P..'' or ``PL''.   Parameter is  photon index  (flux in
photons\,cm$^{-2}$\,s$^{-1}$ is E$^{-(index)}$.  You can now enter a negative
number as the index, and PIMMS will calculate a power law that increases
with increasing energy in photon space.

A power law with high energy cutoff E$^{-index}$ exp[(E$_{cut}-$E)/E$_{efold}$]
can be specified by typing "model pl 1.5 13.5 20.0 1e22" for example ---
this will result in an index of 1.5, cut-off energy of 13.5 and e-folding
energy of 20 keV, with an N$_H$ of 1E22.

\subsubsection*{MODEL --- Plasma}

Can be abbreviated to ``PLA...''.  PIMMS relies on grids of
pre-calculated files, using the APEC (default), mekal, and Raymond-Smith
codes.  PIMMS v4.0 is distributed with APEC and Raymond-Smith models at
59 temperature (logT of 5.60 to 8.50 in an increment of 0.05) times 5
abundance (0.2 to 1.0 in an increment of 0.2), while mekal has a narrower
temperature range starting with logT of 6.0 (51 tempertures to logT=8.50).
The syntax is ``model plasma <kT> <abun> <nh>'' (e.g.,
``model plasma 3.5 0.8 3e20'').
The temperature can be specified in logT as adding the word ``logt'' after
the first numerical parameter, which is then taken to be logT.  Starting in
V4.7, APEC model files calculated using ATOMDB version 2.0.2 in XSPEC
version 12.8.1g on 2013 Dec 9-11, assuming the solar abundances due to
Asplund M et al. 2009, ARAA, 47, 481, are included. Between V4.0 and V4.6b,
APEC files generated using ATOMDB 1.3.1 assuming Anders \& Grevesse were
included. Current mekal model files also assume Anders \& Grevesse solar
abundances, while the solar standard for the Raymond-Smith grid has become
unclear due to passage of time.  PIMMS will select the nearest temperature
and abundance that is supported by the grid in use.

At start-up, the plasma model in use is APEC.  The alternatives can be
selected by issuing the PLASMA command.

\subsubsection*{MODEL --- Gaussian}

Can be abbreviated to ``G''.  This model takes the central energy and physical
width (in keV) as parameters, and optionally also Nh.  A physical width of 0
is allowed, which is interpreted as a delta function (integrations of delta
function is treated appropriately, although it may look incorrect in the
differential form, which is what is saved by the OUTPUT command).

Gaussian is primarily intended as second (etc.) component in addition to a
continuum model, with the same Nh as the primary component.  In such cases,
specify the equivalent width in eV rather than the `relative strength'.

\subsubsection*{MODEL --- External Models}

Other, perhaps more complex, models can be  imported in the form of an Ascii
file containing energy  (keV) vs. flux (photons\,cm$^{-2}$\,s$^{-1}$\,keV$^{-1}$
pairs.    N$_H$
correction is  optional  (i.e., interstellar absorption can be included when
producing the file or done within PIMMS). If the full directory/file name is
not specified, user's current default directory is assumed first, and if not
the models directory is searched.

\subsubsection*{MODEL --- Models directory}

Some external models (see help item under that name) may be kept under the
MODELS subdirectory.   If the file names  and a short description  is also
included in the MODEL.IDX file,  then PIMMS users will be able to see what
is available.

\subsubsection*{MODEL --- Importing from XSPEC}

To import a model from XSPEC, start XSPEC and read, e.g., a template ASCA
SIS pha file (which specifies the SIS response matrix which specifies the
PHA channel  boundary etc.).    Create your model.   Then use IPLOT MODEL
command to plot  the model,  then from within plot  use the WD $<$filename$>$
command to output the model into an Ascii file.  The program XSING within
the MODELS directory  should be used  to convert the  XSPEC output into a
form readable by PIMMS.

\subsection*{PLASMA}

Command syntax: PLASMA apec/mekal/rs \\
Minimum abbreviation: P \\
Example: ``PLASMA MEKAL'' \\
\vspace{0.5 cm}

This command specifies the grid to use when the plasma model is specified.
The default, chosen at start-up, is APEC.  If a plasma model is currently
in use (including as a component of a compound model), the grid is switched
on the fly if at all possible.  If none is being use, this command serves
as a preliminary step for the next MODEL command that does include a plasma
component.

\subsection*{FROM}

Command syntax: FROM $<$mission$>$ [$<$det$>$ [$<$filt$>$]] [$<$lo$>$-$<$hi$>$] \\
\hspace{1.5 cm} FROM FLUX $<$unit$>$ $<$lo$>$-$<$hi$>$ [UNABSORBED] [ANGSTROMS] \\
\hspace{1.5 cm} FROM NORMALIZATION \\
Minimum abbreviation: F \\
Examples: ``FROM EINSTEIN IPC'' ```FROM FLUX PHOTONS 0.5-10'' \\
\vspace{0.5 cm}

This command specifies the default ``instrument'' that the conversion is to take
place from.   This default will be used in  GO (in count rate simulation mode)
or POINT (in image simulation mode) command if not explicitly specified.   See
Missions for details of the available instruments, or try DIRECTORY. Initially
the default is 2.0--10.0 flux in ergs\,cm$^{-2}$\,s$^{-1}$.

\subsection*{INSTRUMENT}

Command syntax: INSTRUMENT $<$mssn$>$ [$<$det$>$ [$<$filt$>$]] [$<$lo$>$-$<$hi$>$] \\
\hspace{1.5 cm} or INSTRUMENT FLUX $<$unit$>$ $<$lo$>$-$<$hi$>$ [UNABSORBED] [ANGSTROMS] \\
Minimum abbreviation: I \\
Examples: ``INST EXOSAT LE LX3'' ``INST FLUX ERGS 1-10 U'' \\
\vspace{0.5 cm}

This command specifies the  ``instrument''  that the conversion is to take place
to.   See Missions for details of the available instruments, or try DIRECTORY.
Initially default is ASCA SIS.

\subsection*{GO}

Minimum abbreviation: G \\
Command syntax: GO $<$input\_rate$>$ [$<$mission$>$ [$<$det$>$ [$<$filt$>$]]] [$<$lo$>$-$<$hi$>$] \\
\hspace{1.5 cm} or GO $<$input\_rate$>$ [FLUX $<$unit$>$ $<$lo$>$-$<$hi$>$ [UNABSORBED]] \\
Examples: ``G 1.0'' ``GO 3.4 EINSTEIN IPC'' \\
\vspace{0.5 cm}

This command actually tells PIMMS to execute the simulation.

Given a source spectrum in the form specified with MODEL,  which produces an
input rate (count rate in the specified instruments or flux) of $<$input\_rate$>$
it GO predicts what the rate  would be for the instrument specified with the
INSTRUMENT command.   Unit of input rate can be specified here,  or else the
default is used (see FROM).

\subsection*{OUTPUT}

Command syntax: OUTPUT $<$filename$>$ $<$loE$>$ $<$hiE$>$ $<$deltaE$>$ \\
Example: ``OUT compoite 0.1 10.0 0.005'' \\
\vspace{0.5 cm}

This command produces an Ascii file containing the current spectral model,
and is intended primarily as a debugging tool for complicated multi-component
models.  Each row consists of energy, total model flux, and flux of each
component if there are more than one.  The energy grid should be specified
using the minimum and maximum values and the increment.

PIMMS currently forces output file names to be all lowercase.

\subsection*{SHOW}

Command syntax: SHOW \\
Minimum abbreviation: SH \\
\vspace{0.5 cm}

Presents a summary of the current defaults on the screen.

\subsection*{DIRECTORY}

Command syntax: DIRECTORY [$<$mission$>$ [$<$detector$>$]] \\
Minimum abbreviation: D \\
Examples: ``DIR'' ``DIRE EXOSAT'' \\
\vspace{0.5 cm}

This command prints, on your screen, the full listing of missions that PIMMS
recognizes.  For explanations and comments, see ``Missions'' in this help.

\subsection*{LOG}

Command syntax: LOG $<$log-file-name$>$ or LOG close \\
Minimum abbreviation: L \\
Example: "LOG crab" \\
\vspace{0.5 cm}

When this command is issued, PIMMS opens a log file (default extension
.log).  Thereafter, screen outputs from PIMMS (except for questions/prompts)
will be copied to the log file.  LOG CLOSE will close the current log file;
the purpose of this command would be to send further output to a separate
log file.  This command will indicate error if (1) a log file is already
open; (2) (on UNIX systems) the specified file already exists; (3) (on
VMS systems) when the specified "log file name" is also a DCL Logical;
and (4) PIMMS failed to open the file for the usual reasons, including
a lack of disk space and file system protection.

PIMMS currently forces output file names to be all lowercase.

\pagebreak

\section*{Appendix B: Older Updates}

\subsection{New in v3.9/3.9a/3.9b/3.9c/3.9d/3.9e/3.9f/3.9g/3.9h/3.9i/3.9j/3.9k}

Version 3.9k incorporates an updated set of effective area curves,
as well as HXD background numbers, for Suzaku, suitable for Cycle 5
propoers.

Version 3.9j incorporates a new set of effective area curves for Chandra,
suitable for use in writing Cycle 11 proposals.

Version 3.9i correctly treats the statistical errors in the Suzaku HXD
background files provided to the users.  Previous versions would have
underestimated the necessary exposure times somewhat.  It also corrects
bugs for Swift XRT grade 0 count rates as well as ASCA count rates within
the typical extraction regions, in cases when a limited energy range is
spacified.  Also updated is the message regarding the instrument-specific
extra information (or the lack thereof) when a limited energy range is
specified.

Version 3.9h includes minor updates to the effective area curves for
Suzaku XIS and HXD instruments.  In addition, PIMMS now outputs
approximate time needed for 3-sigma and 5-sigma detection in standard
HXD energy bands (1 for PIN and 2 for GSO), given the current level
of systematic uncertainties in the background models.

Version 3.9g includes updated effective area curves for XMM-Newton EPIC
MOS cameras, with an added warning from the team regarding MOS1 timing
mode data.  It also includes updated effective area curves for Swift
XRT.

Version 3.9f includes updated effective area curves for Constellation-X
instruments (under the revised mission name, ``conx''), XEUS, INTEGRAL
(ISGRI and JEM-X), and for Swift UVOT (except grisms).

Version 3.9e includes updated effective area curves for the
Chandra instruments for Cycle 10 proposers.

Version 3.9d includes updated effective area curves for Suzaku XIS
and HXD instruments for Cycle 3 proposers.

Version 3.9c includes updated effective area curves for XMM-Newton
EPIC instruments for AO-7 proposers.  The major change is that,
starting with this version, the PN count rates are given for
PATTERN=0 events only, to be conservative.  This will reduce the
predicted count rates by about a third, depending on the spectral
shape.  Pile-up calculation has been adjusted to still use the
total count rates.

Version 3.9b includes a bug fix for energy flux calculations using
redshifted table-based models (including Raymond-Smith).  Effective
area curves for Chandra instruments have been updated for Cycle 9
proposers in v3.9b.

Effective area curves for Suzaku XIS and HXD instruments have been
updated for Cycle 2 proposers in v3.9a.

Effective area curves for XMM-Newton EPIC instruments have been
updated for AO-6 proposers.

One bug, which caused crashes for the command-line version on Linux
platforms, has been fixed in this version.

\subsection{New in v3.8/3.8a}

The Swift/XRT effective area curves for Photon Counting and Windowed Timing
data have been updated in v3.8a.

Previous versions used an old and incorrect deadtime fraction for burst
mode for {\sl XMM-Newton\/} EPIC-pn detector.  This has been corrected.

With this version, PIMMS now calculates Swift/BAT count rates in 4 standard
survey bands, and also includes the background count rates and hence the
exposure time necessary to reach a signal-to-noise ratio of 5.0.

\subsection{New in v3.7/v3.7a}

Suzaku-specific information has been updated in v3.7a, with in-orbit
calibration as of early November, 2005.

This version now provides assumptions regarding the input count rate
(such as extraction region for imaging instruments) for various missions.

This version includes the AO-5 versions of {\sl XMM-Newton\/} EPIC effective area
curves, which differs slightly from the AO-4 versions.  There are no updates
for the RGS effective area.

This version also includes calibration updates of the {\sl RXTE\/} PCA
effective area curves.  The count rate should increase by about 11% compared
to previous versions.  This is deemed minor enough that no impact is expected
in proposals and in observation planning.

PIMMS now uses the post-launch name, {\em Suzaku,\/}
for the former {\em Astro-E2\/}.

This version also includes post-launch calibration of the Swift XRT,
in three data modes (photon counting, photodiode, and windowed timing).

\subsection{New in v3.6/v3.6a/v3.6b/v3.6c}

This version has the new Swift BAT effective area curve for one detector,
to match the default units used in BAT analysis software.  In comparison,
previous versions (v3.6b and earlier) used the effective area for 16384
detectors (the total number fully illuminated by an on-axis source).

V3.6b differs from v3.6a in that it incorporates the new set of Chandra
effective area curves for Cycle 7 proposers.

V3.6a differs from v3.6 in that it includes a new set of effective
area curves for {\em XMM-Newton\/} RGS.

This version includes updated {\em XMM-Newton\/} effective area curves
appropriate for AO-4 proposals.  Note that, starting this version, the
EPIC effective area curves are given for a 15 arcsec radius extraction
region, which is typical for a point source.  On-axis observation is
assumed.

It also corrects a bug in model parser, and now allows specification
of N$_{\rm H}$ for file-based models used as second (etc.) component.

\subsubsection{New in v3.5}

This version includes {\em Astro-E2\/} effective areas, based on best available
calibration data as of 2004 July, and a routine to estimate XRS grade
fractions for bright sources.

\subsubsection{New in v3.4/v3.4a}

V3.4a differs from v3.4 only in having an updated set of Chandra
effective area curves for AO-6 proposers.

Swift effective area curves are newly included; to enable UVOT
count rate predictions, a new UV/optical extinction function has
been added to the code.

\subsubsection{New in v3.3/v3.3a}

PIMMS now allows to convert to/from flux density; it can be in
photons\,cm$^{-2}$s$^{-1}$keV$^{-1}$, ergs\,cm$^{-2}$s$^{-1}$keV$^{-1}$,
photons\,cm$^{-2}$s$^{-1} \AA^{-1}$, or ergs\,cm$^{-2}$s$^{-1} \AA^{-1}$
depending on whether the energy/wavelength is specified as keV or in
$\AA$.

V3.3 also incorporates {\sl Chandra\/} updates for Cycle 5.

V3.3a incorporates {\sl XMM-Newton\/} effective area curves for
AO-3 propoers.

\subsubsection{New in v3.2/3.2a/3.2b/3.2c/3.2d}

Chandra effective area curves have been updated in V3.2d
for Cycle 4 proposers.  In addition, it corrects a minor
bug in {\tt pms\_slmdl.f} (the bug caused PIMMS to run out
of memory unnecessarily when repeatedly reading in file-based
spectral models), and another in {\tt pms\_rarea.f} (this bug
prevented PIMMS from reading effective area cure of the maximum
specified size).

V3.2c corrects one bug introduced in 3.2a (3.2b was an attempt
to correct this, but it was only partially successful); if 3.2a
or 3.2b crashes, obtain either 3.2c or just one corrected file,
{\tt pms\_docrt.f}.  (The XTE HEXTE effective area curves have
also been updated in 3.2c, but this should not have a major effect
on the calculations.)

V3.2a incorporates cosmetic tweaks to eliminate unnecessary warning
messages.

\begin{enumerate}
\item XMM-Newton AO-2 support: Effective area curves for EPIC-MOS, EPIC-PN,
	and RGS have been updated using in-orbit calibrationand.
\item Also includese a new formula to estimate pile-up for EPIC-MOS,
	based on in-orbit data.
\item WARNING: This version does not have an updated pile-up formula for
	EPIC-PN.
\end{enumerate}

\subsubsection{New and updated in v3.1a/3.1b/3.1c}

\begin{enumerate}
\item Bug fix in 3.1c: Multiple component models with two or more
	Raymond-Smith models now work correctly.
\item Chandra AO-3 support, including clarification of first order count rates.
\item Minor clean-ups and bug fixes of v3.0.
\end{enumerate}

\subsubsection{New and updated in v3.0}

\begin{enumerate}
\item Chandra AO-2 support.
\item Preliminary Integral support, including CGRO OSSE effective area curve.
\item Multi-component model capability.
\item Capability to read commands from a file using @$<$filename$>$ syntax.
\end{enumerate}

\subsubsection{New and updated in v2.7}

\begin{enumerate}
\item Latest (1999 June) effective area curves for the ASTRO-E instruments.
\item Now uses XTE PCA count rate per PCU (Proportional Counter Unit), not
	per 5 PCUs.
\item Changed all (except historical) references to `AXAF' to `Chandra'.
\item Enlarged the buffer size for storing external spectral models
	(supplied as an ASCII file) to 65536 lines.
\end{enumerate}

\subsubsection{New and updated in v2.6 through v2.6c}

\begin{enumerate}
\item 1999 January versions of ASTRO-E XRS and XIS responses, including the
  XRS event grade calculations.
\item 1999 January versions of XMM EPIC responses, including the pile-up and
  dead-time calculations.
\item RXTE PCA calibrations appropriate for `Epoch 4' gain setting.
\end{enumerate}

\subsection{New and updated in v2.5}

\begin{enumerate}
\item $\beta$-release support for AXAF ACIS pile-up calculations.
\item Whenever energy range can be/must be specified, wavelength range
	(in $\AA$) can be specified instead.
\item Restrictions on power law index has been eliminated.
\item First preliminary effective area curves for ASTRO-E XRS is included.
\end{enumerate}

\subsection{New and updated in v2.4b}

\begin{enumerate}
\item Updated calibrations of the SAX instruments.
\item Bug-fix for long path/file names
\end{enumerate}

\subsection{New and updated in v2.4}

\begin{enumerate}
\item Updated calibrations of the XTE instruments for AO-3 proposals.
\item New ASCA SIS telemetry limit calculations based on the latest
hot/flickering pixel rates.
\item Pimms now supported on Linux machines with g77, on a beta-test basis.
\item Model normalization can be used as input rate.
\end{enumerate}

\subsection{New and updated in v2.3}

\begin{enumerate}
\item New (in-orbit) calibrations for XTE PCA
and HEXTE instruments.  Use of the new numbers is a requirement for XTE
AO-2 proposals.
\item An optional set of extra Raymond-Smith models, suitable in particular
for EUVE and ROSAT data.
\item The ``Snowden R-band'' effective area curve, courtesy Dr. Richard
West of Leicester.
\item A new command ``LOG'': this allows users to capture PIMMS output in
a log file.
\end{enumerate}

\end{document}
